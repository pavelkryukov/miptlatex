%%%%%%%%%%%%%%%%%%%%%%%%%%%%%%%%%%%%%%%%%%%%%%%%%%%%%%%%%%%%%%%%%%%%%
%% КОНФИГУРАЦИОННЫЙ ФАЙЛ
%%%%%%%%%%%%%%%%%%%%%%%%%%%%%%%%%%%%%%%%%%%%%%%%%%%%%%%%%%%%%%%%%%%%%
%%
%%

%%%%%%%%%%%%%%%%%%%%%%%%%%%%%%%%%%%%%%%%%%%%%%%%%%%%%%%%%%%%%%%%%%%%%
%% Подключаем и настраиваем стилевой пакет KOMA-Sript
\documentclass{scrreprt}
\KOMAoptions{
  fontsize = 12pt, 
  numbers = periodatend, 
  chapteratlists,
  headings = chapterprefix,
  toc = bib,
  captions = tableheading,
  draft
}

%% Настраиваем формат подписей к таблицам и рисункам
\setkomafont{captionlabel}{\bfseries\sffamily}
\setkomafont{caption}{\sffamily}
\renewcommand*{\captionformat}{~}

%% Настраиваем формат сноски
\deffootnote[1.5em]{0em}{1em}{\textsuperscript{\thefootnotemark}\,}

%% Выключаем предупреждение об использовании устаревшего макроса \float@addtolist,
%% который присутствует, например, в пакете listings
\usepackage{scrhack}

%%%%%%%%%%%%%%%%%%%%%%%%%%%%%%%%%%%%%%%%%%%%%%%%%%%%%%%%%%%%%%%%%%%%%
%% Настраиваем кодировкии и язык
\defaulthyphenchar=127
\usepackage{cmap}
\usepackage[T2A]{fontenc}
\usepackage[utf8]{inputenc}
\usepackage[english, greek, russian]{babel}

%% Настраиваем поля и внешний вид страницы
\usepackage{geometry}   
\geometry{
  left = 3cm,
  right = 2cm,
  top = 2cm,
  bottom = 2cm,
  headheight = 1cm,
  headsep = 0.5cm,
  footskip = 1cm
}

%% Устанавливаем полуторный межстрочный интервал
\usepackage{setspace}
\onehalfspacing

%% Включаем отступ у первого абзаца
\usepackage{indentfirst}
%5 Устанавливаем значение абазацного отступа
\setlength{\parindent}{1cm} 

%% Микротипографика (оптическое выравнивание текста)
\usepackage[protrusion, expansion]{microtype}

%% Подключаем и настраиваем пакет для подрисунков
\usepackage[hang, raggedright, small]{subfigure}
% Подрисунки обозначаются русскими буквами
\renewcommand{\thesubfigure}{\emph{\asbuk{subfigure}}}
\makeatletter
  \renewcommand{\p@subfigure}{\thefigure}
  \renewcommand{\@thesubfigure}{\emph{\asbuk{subfigure}})~}
\makeatother

\renewcommand{\thesubtable}{\emph{\asbuk{subtable}}}
\makeatletter
  \renewcommand{\p@subtable}{\thetable}
  \renewcommand{\@thesubtable}{\emph{\asbuk{subtable}})~}
\makeatother

%% Пакет, который не позволяет плавающим объетам выходить за пределы разделов
\usepackage[section]{placeins}
\makeatletter 
% Дополним, чтобы они не выходили и за пределы подразделов
\AtBeginDocument{%
  \expandafter\renewcommand\expandafter\subsection\expandafter
    {\expandafter\@fb@subsecFB\subsection}%
  \newcommand\@fb@subsecFB{\FloatBarrier
  \gdef\@fb@afterHHook{\@fb@topbarrier \gdef\@fb@afterHHook{}}}
  \g@addto@macro\@afterheading{\@fb@afterHHook}
  \gdef\@fb@afterHHook{}
}
\makeatother 

%% Подключаем пакет TikZ и дополнительные пакеты к нему 
\usepackage{tikz}
\makeatletter
\pgfdeclareshape{fflop} {
  \inheritsavedanchors[from=rectangle]
  \inheritanchorborder[from=rectangle]
  \inheritanchor[from=rectangle] {center}
  \inheritanchor[from=rectangle] {north}
  \inheritanchor[from=rectangle] {south}
  \inheritanchor[from=rectangle] {west}
  \inheritanchor[from=rectangle] {east}

  \backgroundpath{
    % Store lower right in xa/ya and upper right in xb/yb
    \southwest \pgf@xa=\pgf@x \pgf@ya=\pgf@y
    \northeast \pgf@xb=\pgf@x \pgf@yb=\pgf@y

    \pgf@xc=.5\wd\pgfnodeparttextbox
    \pgf@yc=\pgf@ya \advance\pgf@yc by 10pt
    % Rectangle
    \pgfpathmoveto{\pgfpoint{\pgf@xa}{\pgf@ya}}
    \pgfpathlineto{\pgfpoint{\pgf@xa}{\pgf@yb}}
    \pgfpathlineto{\pgfpoint{\pgf@xb}{\pgf@yb}}
    \pgfpathlineto{\pgfpoint{\pgf@xb}{\pgf@yb}}
    \pgfpathlineto{\pgfpoint{\pgf@xb}{\pgf@ya}}
    \pgfpathclose
    % Add triangle at the bottom
    \pgfpathmoveto{\pgfpoint{\pgf@xa}{\pgf@ya}}
    \pgfpathlineto{\pgfpoint{\pgf@xc}{\pgf@yc}}
    \pgfpathlineto{\pgfpoint{\pgf@xb}{\pgf@ya}}
  }
}
\makeatother

\usetikzlibrary{
  shapes,
  arrows,
  intersections,
  positioning,
  fit
}
\tikzset{
  font = \small\sffamily,
  inner sep = 5pt,
  node distance = 20pt,
  trapezium stretches body,
  every fit/.append style = text badly centered,
  pre/.style = {<-, shorten <=1pt, >=stealth'},
  post/.style = {->, shorten >=1pt, >=stealth'},
  block/.style = {
    draw = #1!80,
    fill = #1!20,
    align = center,
    text badly centered
  },
  block/.default = blue,
  flop/.style = {
    block = gray,
    shape = fflop,
    minimum width = .5cm,
    minimum height = 1.5cm
  },
  port/.style = {
    block = red,
    font = \rmfamily,
    text width = .3cm,
    minimum height = .5cm
  },
  mux/.style = {
    block = gray,
    trapezium,
    shape border uses incircle,
    shape border rotate = -90,
    minimum width = 1cm
  },
  demux/.style = {
    mux,
    shape border rotate = 90
  },
  FSM/.style = {
    block = red,
    rounded corners,
    minimum height = 1cm,
    minimum width = 1.5cm
  },
  label/.style = {
    font = \scriptsize\sffamily,
    inner sep = 1pt
  },
  exchange/.style = {
    <->,
    shorten <=1pt,
    shorten >=1pt,
    >=fast cap,
    line width = 5pt
  }, 
  arith/.style = {
    block,
    minimum width =  1.5cm,
    minimum height = 1.5cm
  },
  module/.style = {
    draw = black!50,
    fill = yellow!20,
    rounded corners,
    inner sep = 10pt,
    dashed
  }
}
% Настраиваем слои
\pgfdeclarelayer{background}
\pgfdeclarelayer{foreground}
\pgfsetlayers{background,main,foreground}

%% Подключаем и настраиваем пакет для кастомизированных списков
\usepackage{enumitem}
% Регистрируем новый счетчик для списков с русскими буквами
\makeatletter
  \AddEnumerateCounter{\asbuk}{\@asbuk}{\cyrzh}
  \AddEnumerateCounter{\Asbuk}{\@Asbuk}{\CYRZH}
\makeatother
% Убирем промежутки до и после списка и настраиваем выравнивание
\setlist{leftmargin = *, nolistsep}
% Для списков первого уровня устанавливаем отсуп, равный абзацному
\setlist[1]{labelindent = \parindent}
% Перечислением по умолчанием делаем счет вида 1)
\setenumerate{label = \arabic*)}
% Символом для списков по умолчанию делаем тире
\setitemize{label = \textemdash}

%% Настраиваем пакеты для работы с таблицами
\usepackage{longtable, array, colortbl}
\usepackage{booktabs}
\usepackage{multirow}
\newcolumntype{L}{>{$}l<{$}}
\newcolumntype{C}{>{$}c<{$}}

% Математическая (math) колонка
\newcolumntype{m}[1]{>{\raggedright\arraybackslash $}p{#1}<{$}}
% Текстовая (text) колонка
\newcolumntype{t}[1]{>{\raggedright\arraybackslash}p{#1}<{}}
\renewcommand{\arraystretch}{1.3}
\renewcommand{\arrayrulewidth}{0.6pt}
% Устанавливаем цвет для ячеек шапки таблицы
\colorlet{tableheadcolor}{gray!25}

\usepackage{lscape, pdflscape} 
\usepackage{amsmath, amssymb, ifthen, calc}

\usepackage[group-minimum-digits = 4]{siunitx}

%% Подключаем пакет xspace
\usepackage{xspace}
% Чтобы не ставил пробел перед кавычками и после кавычек
\xspaceaddexceptions{<< >> ,, ``}

%% Подключаем пакет дат
\usepackage{datetime}
% Устанавливаем формат даты
\ddmmyydate

%% Подключаем пакет для отображения листингов программ
\usepackage{listings}

\renewcommand{\lstlistingname}{Листинг}
% Ставим точку после номера перед названием
\makeatletter
  \AtBeginDocument{\renewcommand \thelstlisting{\thechapter.\@arabic\c@lstlisting\autodot}}
\makeatother
% Настраиваем стиль по-умолчанию


\DeclareNewTOC[
  type = codebox,
  types = codeboxes,
  float,
  floatpos = h,
  floattype = 4,
  nonfloat,
  name = Листинг,
  listname = {Список листингов}
]{lor}
\setuptoc{lor}{chapteratlist}
\makeatletter
  \AtBeginDocument{\renewcommand \thecodebox{\thechapter.\@arabic\c@codebox}}
\makeatother

%% Подключаем пакет для форматирования алгоритмов
\usepackage[noend]{algorithmic}
\algsetup{
  linenosize = \tiny,
  linenodelimiter = \ 
}
% Русифицируем ключевые слова
\def\algorithmicrequire{\textbf{Вход:}}
\def\algorithmicensure{\textbf{Выход:}}
\def\algorithmicif{\textbf{если}}
\def\algorithmicthen{\textbf{то}}
\def\algorithmicelse{\textbf{иначе}}
\def\algorithmicelsif{\textbf{иначе если}}
\def\algorithmicfor{\textbf{для}}
\def\algorithmicforall{\textbf{для всех}}
\def\algorithmicdo{}
\def\algorithmicwhile{\textbf{пока}}
\def\algorithmicrepeat{\textbf{повторять}}
\def\algorithmicuntil{\textbf{пока}}
\def\algorithmicloop{\textbf{цикл}}
\def\algorithmiccomment#1{\quad/\!/ \textit{#1}}

%% Подсключаем пакет для отображения гиперссылок
\usepackage{hyperref}
\urlstyle{sf}

% Делаем правильный кернинг двойного слеша
\makeatletter
\g@addto@macro{\Url@acthash}{\Url@Edit\Url@String{//}{/\kern-0.2em/}}
\makeatother

% Русский математический рукописный шрифт
\usepackage{mathrsfs}

% Готический шрифт
\usepackage{yfonts}

% Диаграммы
\usepackage{pb-diagram}

% Для многострочных комментариев
\usepackage{verbatim}

%% Делаем подсчет объектов (страниц, рисунков, таблиц, биб. записей) в документе
\usepackage{totcount, etoolbox}
\regtotcounter{page}
\regtotcounter{figure}
\regtotcounter{table}
\regtotcounter{codebox}
\regtotcounter{enumiv}

\newcounter{totfigures}
\newcounter{tottables}
\newcounter{totlistings}

\providecommand\totfig{} 
\providecommand\tottab{}
\providecommand\totlst{}

\makeatletter
\AtEndDocument{%
  \addtocounter{totfigures}{\value{figure}}%
  \addtocounter{tottables}{\value{table}}%
  \addtocounter{totlistings}{\value{codebox}}%
  \immediate\write\@mainaux{%
    \string\gdef\string\totfig{\number\value{totfigures}}%
    \string\gdef\string\tottab{\number\value{tottables}}%
    \string\gdef\string\totlst{\number\value{totlistings}}%
  }%
}
\makeatother

\pretocmd{\chapter}{\addtocounter{totfigures}{\value{figure}}\setcounter{figure}{0}}{}{}
\pretocmd{\chapter}{\addtocounter{tottables}{\value{table}}\setcounter{table}{0}}{}{}
\pretocmd{\chapter}{\addtocounter{totlistings}{\value{codebox}}\setcounter{codebox}{0}}{}{} 

% Пакет для создания алфавитных указателей
\usepackage{makeidx}

% Пакет для эпиграфов
\usepackage{epigraph}

% Пакет для ссылки на последнюю страницу
\usepackage{lastpage}

% Стрелки в тексте вне формул
\usepackage{textcomp}

% Subsection без нумерации
\makeatletter
\newcommand*\addsubsec{\secdef\@addsubsec\@saddsubsec}
\newcommand*{\@addsubsec}{}
\def\@addsubsec[#1]#2{\subsection*{#2}\addcontentsline{toc}{subsection}{#1}
  \if@twoside\ifx\@mkboth\markboth\markright{#1}\fi\fi
}
\newcommand*{\@saddsubsec}[1]{\subsection*{#1}\@mkboth{}{}}
\makeatother

% Subsubsection без нумерации
\makeatletter
\newcommand*\addsubsubsec{\secdef\@addsubsubsec\@saddsubsubsec}
\newcommand*{\@addsubsubsec}{}
\def\@addsubsubsec[#1]#2{\subsubsection*{#2}\addcontentsline{toc}{subsubsection}{#1}
  \if@twoside\ifx\@mkboth\markboth\markright{#1}\fi\fi
}
\newcommand*{\@saddsubsubsec}[1]{\subsubsection*{#1}\@mkboth{}{}}
\makeatother

%%%%%%%%%%%%%%%%%%%%%%%%%%%%%%%%%%%%%%%%%%%%%%%%%%%%%%%%%%%%%%%%%%%%%
%% Делаем правильные переносы в математических формулах, когда
%% знак операции повторяется на новой строке 
%%
%% Код основан на коде пакета 'russmath', автор: leontiev@aport.ru
%%
\newif\ifBrkFlag
\def\allowmathbreak{\global\BrkFlagtrue\global\relpenalty=10000\global\binoppenalty=10000}
\def\cancelmathbreak{\global\BrkFlagfalse\global\relpenalty=500\global\binoppenalty=700}
\allowmathbreak

\makeatletter
\def\m@th{\mathsurround=0pt}
\def\@thick{\hbox{\m@th$\mskip\thickmuskip$}}
\def\@med{\hbox{\m@th$\mskip\medmuskip$}}

\def\brel#1{\ifBrkFlag\discretionary{\@thick\hbox{\m@th$#1$}}{}{}\fi #1}
\def\bbin#1{\ifBrkFlag\discretionary{\@med\hbox{\m@th$#1$}}{}{}\fi #1}
\makeatother