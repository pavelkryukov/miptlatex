%%%%%%%%%%%%%%%%%%%%%%%%%%%%%%%%%%%%%%%%%%%%%%%%%%%%%%%%%%%%%%%%%%%%%
%% МАКРОСЫ
%%%%%%%%%%%%%%%%%%%%%%%%%%%%%%%%%%%%%%%%%%%%%%%%%%%%%%%%%%%%%%%%%%%%%
%%
%%

%% Текстовые сокращения
% Exapmli gratia (eg.) = Например (напр.)
\newcommand{\eg}{например,\xspace}
\newcommand{\Eg}{Например,\xspace}
% Et cetera (etc.) = И так далее (и т.д.)
\renewcommand{\etc}{и~т.\,д.\xspace}
% Id est (i.e.) = То есть (т.е.)
\newcommand{\ie}{т.\,е.\xspace}
\newcommand{\Ie}{Т.\,е.\xspace}
% Et alii (et al.) = И другие (и др.)
\newcommand{\etal}{и~др.\xspace}
%% (стр. 99)
\newcommand{\pagerefrus}[1]{(стр.\,\pageref{#1})}

%% Макросы для заглушек
\newcommand{\fixme}{FIXME}
\newcommand{\unref}{[???]}

\newcommand{\kips}{$\text{кк}/\text{с}$\xspace}

%% Макросы
%% Для выделения текста
\newcommand{\strong}[1]{\textbf{#1}}
% Для вставки иностранных слов и словосочетаний
\newcommand{\foreigne}[1]{\mbox{англ.}~\emph{\foreignlanguage{english}{#1}}} % английский
\newcommand{\foreigng}[1]{\mbox{греч.}~\foreignlanguage{greek}{#1}} % греческий
\newcommand{\foreignl}[1]{\mbox{лат.}~\emph{\foreignlanguage{latin}{#1}}} % латинский
\newcommand{\foreignfrak}[1]{\mbox{нем.}~\textfrak{#1}} % немецкий фрактурный
%% Для выделения ключевых слов (терминов)
\newcommand{\keyword}[2][russian]{\textsf{\foreignlanguage{#1}{#2}}}
\newcommand{\keyworde}[2]{\textsf{#1} (\foreigne{#2})}
\newcommand{\keywordg}[2]{\textsf{#1} (\foreigng{#2})}
\newcommand{\keywordl}[2]{\textsf{#1} (\foreignl{#2})}
\newcommand{\keywordfrak}[2]{\textsf{#1} (\foreignfrak{#2})}

%% Для вставки сокращений
\newcommand{\gap}{\ensuremath{\langle\ldots\rangle}\xspace}
%% Для вставки ссылки на рисунки и таблицы
\newcommand{\seefig}[1]{см.~рис.~\ref{#1}}
\newcommand{\fig}[1]{рис.~\ref{#1}}
\newcommand{\seetab}[1]{см.~табл.~\ref{#1}}
\newcommand{\tab}[1]{табл.~\ref{#1}}

%% Окружение для вставки длинных цитат на различных языках
\newenvironment{mquote}[1][russian]{%
  \bigskip%
  \begin{addmargin}{1.1\parindent}%
  \setlength{\parindent}{0pt}%
  \setlength{\parskip}{\smallskipamount}%
  \selectlanguage{#1}%
  \sffamily\small%
}{
  \end{addmargin}%
  \bigskip%
}

%% Более красивый вариант написания названия языка C++
\newcommand{\CPP}{C\kern-.05em\raise.23ex\hbox{+\kern-.05em+}\xspace}
\newcommand{\CCPP}{C/C\kern-.05em\raise.23ex\hbox{+\kern-.05em+}\xspace}

\newcommand{\midtilde}{\raise-0.5ex\hbox{\textasciitilde}}

\renewcommand{\land}{\bbin{\mathbin{\&}}}
\renewcommand{\lor}{\bbin{\mathbin{\vee}}}
%% Импликация
\newcommand{\impl}{\bbin{\mathbin{\Rightarrow}}}
%% Эквивалентость
\newcommand{\equi}{\bbin{\mathbin{\Leftrightarrow}}}
%% Обратная импликация (converse implication)
\newcommand{\coimpl}{\bbin{\mathbin{\Leftarrow}}}

\newcommand{\bin}{\brel{\in}}
\newcommand{\nin}{\brel{\notin}}
%% Значок пустого множество (empty set)
\newcommand{\es}{\varnothing}
%% Равенство 
\newcommand{\eq}{\brel{=}}
%% Неравенство 
\renewcommand{\neq}{\brel{\ne}}
%% Больше или равно (greater or equal)
\renewcommand{\ge}{\brel{>}}
\renewcommand{\geq}{\brel{\geqslant}}
%% Меньше или равно (less or equal)
\renewcommand{\le}{\brel{<}}
\renewcommand{\leq}{\brel{\leqslant}}
%% Строгое подмножество
\newcommand{\subs}{\brel{\subset}}
%% Подмножество
\newcommand{\subseq}{\brel{\subseteq}}
%% Разность (without)
\newcommand{\wo}{\bbin{\setminus}}
%% Симметрическая разность
\newcommand{\swo}{\bbin{\mathbin{\div}}}
%% Декартово произведение
\newcommand{\dec}{\bbin{\times}}
%% Запретитель разрыва страницы между двоеточием и списком
\makeatletter
\newcommand{\listnopagebreak}{\par\nobreak\@afterheading}
\makeatother
%% До и после нашей эры
\newcommand{\bc}{до\,н.\,э.}
\newcommand{\ad}{н.\,э.}
