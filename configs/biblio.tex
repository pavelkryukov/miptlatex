%% Пакет natbib используется для переопределения команды \bibsection,
%% чтобы отдельные списки литературы, вводимые пакетом multibib, 
%% начинались не новой главой (\chapter), а новым разделом (\section)
%% Также этот пакет реализует функционал пакета cite (sort&compress)
\usepackage[numbers, square, comma, sort&compress]{natbib}
% Меняем формат нумерации списка литературы с [1] на 1.
\makeatletter\renewcommand\@biblabel[1]{#1.}\makeatother
% Меняем главу на раздел
\renewcommand\bibsection{\addsec*{\refname}}

%% Подключаем пакет multibib, чтобы разбить список литературы на части
\usepackage{multibib}
% Определяем названия и коды для разделов списка литературы
\newcites{book}{Монографии}
\newcites{article}{Статьи}
\newcites{diss}{Диссертации и авторефераты}
\newcites{online}{Онлайн-материалы}
\newcites{patent}{Патенты}

%% Донастраиваем стиль списка литературы ГОСТ 7.0.5-2008
\providecommand*{\BibEmph}[1]{\emph{#1}}
