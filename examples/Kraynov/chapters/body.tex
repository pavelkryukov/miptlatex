\strong{Теплопроводность}~"--- явление непосредственного молекулярного переноса энергии из~мест с~более высокой температурой в~места с~более низкой температурой, не~связанного с~макроскопическим движением и~происходящем также~и~в~неподвижной жидкости \cite[с.~269]{landauVI}.

Пусть $\mathbf{q}$ есть плотность потока тепла, переносимого посредством теплопроводности.
В~случае небольшого градиента температур в~жидкости можно считать поток тепла пропорциональным величине градиента температур:
$$\mathbf{q} = - \varkappa\nabla{T}.$$

Величина~$\varkappa$ называется теплопроводностью жидкости.
Так~как $\mathbf{q}$ и~$\nabla{T}$ очевидно должны иметь противоположные направления, то~$\varkappa > 0$.

C~учётом вязкости и~теплопроводности полная плотность потока энергии в~жидкости записывается в~виде
$$\rho\mathbf{v}\left(\frac{v^2}{2} + w\right)
- \left(\mathbf{v}\boldsymbol{\sigma^\prime}\right)
- \varkappa\nabla{T}.$$

Соответственно, закон сохранения энергии будет выглядеть так:
$$
\frac{\partial}{\partial{t}}\left(\frac{\rho{}v^2}{2} + \rho\varepsilon\right) =
- \operatorname{div}{\left(\rho\mathbf{v}\left(\frac{v^2}{2} + w\right)
- \left(\mathbf{v}\boldsymbol{\sigma^\prime}\right)
- \varkappa\nabla{T}\right)}.$$

Преобразовав, используя уравнения движения жидкости, получим \emph{общее уравнение переноса тепла} \citebook[c.~271]{landauVI}
$$\rho{}T\left(\frac{\partial{s}}{\partial{t}} + \mathbf{v}\nabla{s}\right) =
\sigma^{\prime}_{ik}\frac{\partial{v_i}}{\partial{x_k}} + \operatorname{div}{\left(\varkappa\nabla{T}\right)}.
$$

Это уравнение можно существенно упростить, если~рассматривать \strong{несжимаемую жидкость}.
Заметим, что~в~этой модели плотность несжимаемой жидкости не~является постоянной, так~как она зависит от~температуры, и~более корректно принять за~постоянную величину давление.
В~этом случае производные энтропии можно будет переписать, используя производные при~постоянном давлении:
$$\begin{aligned}
    T\frac{\partial{s}}{\partial{t}} &=
	T\left(\frac{\partial{s}}{\partial{T}}\right)_p\frac{\partial{T}}{\partial{t}} &=
	c_p\frac{\partial{T}}{\partial{t}}, \\
    T\nabla{s} &=
    T\left(\frac{\partial{s}}{\partial{T}}\right)_p\nabla{T} &=
    c_p\nabla{T}. \\
\end{aligned}$$

При подстановке в~общее уравнение переноса тепла получим
$$\rho{}c_p\left(\frac{\partial{T}}{\partial{t}} + \mathbf{v}\nabla{T}\right) =
\sigma^{\prime}_{ik}\frac{\partial{v_i}}{\partial{x_k}} + \operatorname{div}{\left(\varkappa\nabla{T}\right)}
.$$

Величина $\varkappa$, вообще~говоря, является функцией давления и~температуры, но~так~как давление постоянно, а~разности температур мы~будем считать малыми, то~$\varkappa$ можно вынести за~знак дивергенции 
$$\operatorname{div}{\left(\varkappa\nabla{T}\right)} = \varkappa\Delta{T}$$

Подстановка в~исходное уравнение и~деление его на~$\rho{}c_p$ даст
$$\frac{\partial{T}}{\partial{t}} + \mathbf{v}\nabla{T} =
\frac{1}{\rho{}c_p}{\sigma^{\prime}_{ik}\frac{\partial{v_i}}{\partial{x_k}} + \frac{\varkappa}{\rho{}c_p}\Delta{T}}
.$$

Введём величину \emph{температуропроводности} $\chi = \frac{\varkappa}{\rho{}c_p}$, это даст окончательный вид уравнения
$$\frac{\partial{T}}{\partial{t}} + \mathbf{v}\nabla{T} =
\frac{1}{\rho{}c_p}{\sigma^{\prime}_{ik}\frac{\partial{v_i}}{\partial{x_k}} + \chi\Delta{T}}
,$$
которое очень просто выглядит в~случае неподвижной жидкости ($\mathbf{v} = \mathbf{0}$):

$$\frac{\partial{T}}{\partial{t}} = \chi\Delta{T}.$$

Отметим, что~это \emph{уравнение~Фурье} имеет, помимо простого вида, простой физический смысл.
Количество тепла, поглощающееся в~единичном объёме тепла за~единицу времени, равно $\rho{}c_p\frac{\partial{T}}{\partial{t}}$, так~как у~неподвижной жидкости постоянно давление.
С~другой стороны, существует и~поток тепла за~счёт теплопроводности, равный $\mathbf{q} = - \varkappa\nabla{T}$.
Для отнесения потока к~единичному объёму возьмем оператор дивергенции, и~тогда закон сохранения энергии для~единичного объёма даст исходное соотношение:
$$0 = \rho{}c_p\frac{\partial{T}}{\partial{t}} + \operatorname{div}{\left(-\varkappa\nabla{T}\right)} =
\rho{}c_p\left(\frac{\partial{T}}{\partial{t}} - \chi\Delta{T}\right).$$

К~сожалению, на~практике условие неподвижности жидкости выполняется плохо, поскольку уже при~малых градиентах температуры наблюдается явление конвекции.
Тем не~менее, уравнение Фурье можно успешно применять для~вязких жидкостей, в~которых конвекция затруднена; в~случаях, когда градиент температуры направлен противоположно силе тяжести, которая компенсирует конвекцию; справедливо оно и~для твёрдых тел.

Рассмотрим некоторые важные частные случаи.
Если распределение температуры постоянно во~времени за~счёт внешних источников тепла, то~левая часть уравнения есть ноль, и~распределение температуры описывается уравнением Лапласа $\Delta{T} = 0$.
Для~случая постоянного распределения температуры достаточно просто решать уравнения, считая $\varkappa$ непостоянным, потому~что~уравнение имеет вид $\operatorname{div}{\left(\varkappa\nabla{T}\right)} = 0$.

Часто бывает нужно изучить дополнительное влияние внешних источников тепла.
Зададим их функцией $Q(\vec{x}, t)$, определяющей количество тепла в~единице объёма за~единицу времени.
Уравнение теплопроводности примет вид

$$Q(\vec{x}, t) = \rho{}c_p\frac{\partial{T}}{\partial{t}} -\varkappa\Delta{T}.$$

Рассмотрим теплопроводность на~границе двух сред, запишем граничные условия.
Во-первых, это равенство температур, а во-вторых, равенство входящего из~одной среды потока тепла входящему в~другую
$$\varkappa_1\nabla{T_1}d\mathbf{f} = \varkappa_2\nabla{T_2}d\mathbf{f},$$
где $d\mathbf{f}$ есть элемент поверхности соприкосновения сред.
Вспомнив, что скалярное произведение градиента на~вектор есть производная по~направлению вектора, в~данном случае вектора нормали к~поверхности

$$\nabla{T}d\mathbf{f} = \frac{\partial{T}}{\partial{\mathbf{n}}}df,$$
условия можно записать в~виде
$$\left\{
    \begin{aligned}
	   T_1 &= T_2 \\
	   \varkappa_1\frac{\partial{T_1}}{\partial{\mathbf{n}}} &= \varkappa_2\frac{\partial{T_2}}{\partial{\mathbf{n}}} \\
	\end{aligned}
   \right.
$$

На~границе двух сред могут находится источники тепла, тогда второе условие примет вид:
$$\varkappa_1\frac{\partial{T_1}}{\partial{\mathbf{n}}} - \varkappa_2\frac{\partial{T_2}}{\partial{\mathbf{n}}} = Q(f, t).$$

Часто в~задачах, особенно химических, функция источника тепла зависит не~от координат, а~напрямую от~температуры; таким образом можно описывать эндо- и~экзотермические реакции, интенсивность которых зависит от~температуры.
Функция $Q(T)$ может очень быстро возрастать, например, по~экспоненте $Q = Q_0\exp\left({\alpha(T-T_0)}\right)$ и~в~таком случае стационарное распределение температуры невозможно.
Более того, рост появляющегося в~системе тепла приводит к~дальнейшему её разогреву и~всё~большей отдаче тепла, происходит \emph{тепловой взрыв}.
Условие его возникновения в~слое можно получить, рассмотрев, когда~имеет решение уравнение
$$\left\{
    \begin{aligned}
       \varkappa\frac{d^2T}{dx^2} &= -Q_0\exp\left({\alpha(T-T_0)}\right) \\
       T(x = 0) &= T(x = 2l) = T_0. \\
	\end{aligned}
   \right.
$$

После введения безразмерных величин
$$\tau = \alpha(T - T_0), \xi = \frac{x}{l}, \lambda = \frac{Q_0\alpha{}l^2}{\varkappa},$$
$$\tau(0) = \tau(2) = 0,$$
уравнение примет более удобный для~интегрирования вид:
$$\begin{aligned}
    \tau'' &= - \lambda{}e^{\tau} \\ 
    2\tau'\tau'' &= - 2\lambda{}e^{\tau}\tau' \\
    \tau'^2 &= 2\lambda{}(e^{\tau_0} - e^{\tau}) \\
    \frac{d\tau}{d\xi} &= \sqrt{2\lambda{}(e^{\tau_0} - e^{\tau})} \\
    d\xi &= \frac{1}{\sqrt{2\lambda}}\frac{d\tau}{\sqrt{e^{\tau_0} - e^{\tau}}}
\end{aligned}.$$

$\tau_0$~"--- постоянная интегрирования, задающее максимальное значение $\tau$, которое, исходя из~симметрии задачи, достигается при~$\xi = 1$, поэтому интегрирование по~$\tau$ будем проводить до~$\tau_0$
$$\int\limits_0^1d\xi = \frac{1}{\sqrt{2\lambda}}\int\limits_0^{\tau_0}\frac{d\tau}{\sqrt{e^{\tau_0} - e^{\tau}}},$$
$$\sqrt{2\lambda} = 2e^{-\tau_0/2}\operatorname{Arch}{e^{\tau_0/2}}.$$

$\lambda(\tau_0)$ имеет максимум $\lambda_m$ при~определённом значении $\tau_m$, если~$\lambda > \lambda_m$, то~стационарного распределения температур не~существует.

\setbiblabelwidth{99}
\bibliographystyle{../../configs/gost2008s}
\bibliography{../../global}