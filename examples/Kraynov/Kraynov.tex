% Общая конфигурация
%%%%%%%%%%%%%%%%%%%%%%%%%%%%%%%%%%%%%%%%%%%%%%%%%%%%%%%%%%%%%%%%%%%%%
%% ���������������� ����
%%%%%%%%%%%%%%%%%%%%%%%%%%%%%%%%%%%%%%%%%%%%%%%%%%%%%%%%%%%%%%%%%%%%%
%%
%%

%%%%%%%%%%%%%%%%%%%%%%%%%%%%%%%%%%%%%%%%%%%%%%%%%%%%%%%%%%%%%%%%%%%%%
%% ���������� � ����������� �������� ����� KOMA-Sript
\documentclass{scrreprt}
\KOMAoptions{
  fontsize = 12pt, 
  numbers = periodatend, 
  chapteratlists,
  headings = chapterprefix,
  toc = bib,
  captions = tableheading,
  draft
}

%% ����������� ������ �������� � �������� � ��������
\setkomafont{captionlabel}{\bfseries\sffamily}
\setkomafont{caption}{\sffamily}
\renewcommand*{\captionformat}{~}

%% ����������� ������ ������
\deffootnote[1.5em]{0em}{1em}{\textsuperscript{\thefootnotemark}\,}

%% ��������� �������������� �� ������������� ����������� ������� \float@addtolist,
%% ������� ������������, ��������, � ������ listings
\usepackage{scrhack}

%%%%%%%%%%%%%%%%%%%%%%%%%%%%%%%%%%%%%%%%%%%%%%%%%%%%%%%%%%%%%%%%%%%%%
%% ����������� ���������� � ����
\defaulthyphenchar=127
\usepackage{cmap}
\usepackage[T2A]{fontenc}
\usepackage[cp1251]{inputenc}
\usepackage[english, greek, russian]{babel}

%% ����������� ���� � ������� ��� ��������
\usepackage{geometry}   
\geometry{
  left = 3cm,
  right = 2cm,
  top = 2cm,
  bottom = 2cm,
  headheight = 1cm,
  headsep = 0.5cm,
  footskip = 1cm
}

%% ������������� ���������� ����������� ��������
\usepackage{setspace}
\onehalfspacing

%% �������� ������ � ������� ������
\usepackage{indentfirst}
%5 ������������� �������� ���������� �������
\setlength{\parindent}{1cm} 

%% ���������������� (���������� ������������ ������)
\usepackage[protrusion, expansion]{microtype}

%% ���������� � ����������� ����� ��� �����������
\usepackage[hang, raggedright, small]{subfigure}
% ���������� ������������ �������� �������
\renewcommand{\thesubfigure}{\emph{\asbuk{subfigure}}}
\makeatletter
  \renewcommand{\p@subfigure}{\thefigure}
  \renewcommand{\@thesubfigure}{\emph{\asbuk{subfigure}})~}
\makeatother

\renewcommand{\thesubtable}{\emph{\asbuk{subtable}}}
\makeatletter
  \renewcommand{\p@subtable}{\thetable}
  \renewcommand{\@thesubtable}{\emph{\asbuk{subtable}})~}
\makeatother

%% �����, ������� �� ��������� ��������� ������� �������� �� ������� ��������
\usepackage[section]{placeins}
\makeatletter 
% ��������, ����� ��� �� �������� � �� ������� �����������
\AtBeginDocument{%
  \expandafter\renewcommand\expandafter\subsection\expandafter
    {\expandafter\@fb@subsecFB\subsection}%
  \newcommand\@fb@subsecFB{\FloatBarrier
  \gdef\@fb@afterHHook{\@fb@topbarrier \gdef\@fb@afterHHook{}}}
  \g@addto@macro\@afterheading{\@fb@afterHHook}
  \gdef\@fb@afterHHook{}
}
\makeatother 

%% ���������� ����� TikZ � �������������� ������ � ���� 
\usepackage{tikz}
\makeatletter
\pgfdeclareshape{fflop} {
  \inheritsavedanchors[from=rectangle]
  \inheritanchorborder[from=rectangle]
  \inheritanchor[from=rectangle] {center}
  \inheritanchor[from=rectangle] {north}
  \inheritanchor[from=rectangle] {south}
  \inheritanchor[from=rectangle] {west}
  \inheritanchor[from=rectangle] {east}

  \backgroundpath{
    % Store lower right in xa/ya and upper right in xb/yb
    \southwest \pgf@xa=\pgf@x \pgf@ya=\pgf@y
    \northeast \pgf@xb=\pgf@x \pgf@yb=\pgf@y

    \pgf@xc=.5\wd\pgfnodeparttextbox
    \pgf@yc=\pgf@ya \advance\pgf@yc by 10pt
    % Rectangle
    \pgfpathmoveto{\pgfpoint{\pgf@xa}{\pgf@ya}}
    \pgfpathlineto{\pgfpoint{\pgf@xa}{\pgf@yb}}
    \pgfpathlineto{\pgfpoint{\pgf@xb}{\pgf@yb}}
    \pgfpathlineto{\pgfpoint{\pgf@xb}{\pgf@yb}}
    \pgfpathlineto{\pgfpoint{\pgf@xb}{\pgf@ya}}
    \pgfpathclose
    % Add triangle at the bottom
    \pgfpathmoveto{\pgfpoint{\pgf@xa}{\pgf@ya}}
    \pgfpathlineto{\pgfpoint{\pgf@xc}{\pgf@yc}}
    \pgfpathlineto{\pgfpoint{\pgf@xb}{\pgf@ya}}
  }
}
\makeatother

\usetikzlibrary{
  shapes,
  arrows,
  intersections,
  positioning,
  fit
}
\tikzset{
  font = \small\sffamily,
  inner sep = 5pt,
  node distance = 20pt,
  trapezium stretches body,
  every fit/.append style = text badly centered,
  pre/.style = {<-, shorten <=1pt, >=stealth'},
  post/.style = {->, shorten >=1pt, >=stealth'},
  block/.style = {
    draw = #1!80,
    fill = #1!20,
    align = center,
    text badly centered
  },
  block/.default = blue,
  flop/.style = {
    block = gray,
    shape = fflop,
    minimum width = .5cm,
    minimum height = 1.5cm
  },
  port/.style = {
    block = red,
    font = \rmfamily,
    text width = .3cm,
    minimum height = .5cm
  },
  mux/.style = {
    block = gray,
    trapezium,
    shape border uses incircle,
    shape border rotate = -90,
    minimum width = 1cm
  },
  demux/.style = {
    mux,
    shape border rotate = 90
  },
  FSM/.style = {
    block = red,
    rounded corners,
    minimum height = 1cm,
    minimum width = 1.5cm
  },
  label/.style = {
    font = \scriptsize\sffamily,
    inner sep = 1pt
  },
  exchange/.style = {
    <->,
    shorten <=1pt,
    shorten >=1pt,
    >=fast cap,
    line width = 5pt
  }, 
  arith/.style = {
    block,
    minimum width =  1.5cm,
    minimum height = 1.5cm
  },
  module/.style = {
    draw = black!50,
    fill = yellow!20,
    rounded corners,
    inner sep = 10pt,
    dashed
  }
}
% ����������� ����
\pgfdeclarelayer{background}
\pgfdeclarelayer{foreground}
\pgfsetlayers{background,main,foreground}

%% ���������� � ����������� ����� ��� ����������������� �������
\usepackage{enumitem}
% ������������ ����� ������� ��� ������� � �������� �������
\makeatletter
  \AddEnumerateCounter{\asbuk}{\@asbuk}{\cyrzh}
  \AddEnumerateCounter{\Asbuk}{\@Asbuk}{\CYRZH}
\makeatother
% ������ ���������� �� � ����� ������ � ����������� ������������
\setlist{leftmargin = *, nolistsep}
% ��� ������� ������� ������ ������������� �����, ������ ���������
\setlist[1]{labelindent = \parindent}
% ������������� �� ���������� ������ ���� ���� 1)
\setenumerate{label = \arabic*)}
% �������� ��� ������� �� ��������� ������ ����
\setitemize{label = \textemdash}

%% ����������� ������ ��� ������ � ���������
\usepackage{longtable, array, colortbl}
\usepackage{booktabs}
\usepackage{multirow}
\newcolumntype{L}{>{$}l<{$}}
\newcolumntype{C}{>{$}c<{$}}

% �������������� (math) �������
\newcolumntype{m}[1]{>{\raggedright\arraybackslash $}p{#1}<{$}}
% ��������� (text) �������
\newcolumntype{t}[1]{>{\raggedright\arraybackslash}p{#1}<{}}
\renewcommand{\arraystretch}{1.3}
\renewcommand{\arrayrulewidth}{0.6pt}
% ������������� ���� ��� ����� ����� �������
\colorlet{tableheadcolor}{gray!25}

\usepackage{lscape, pdflscape} 
\usepackage{amsmath, amssymb, ifthen, calc}

\usepackage[group-minimum-digits = 4]{siunitx}

%% ���������� ����� xspace
\usepackage{xspace}
% ����� �� ������ ������ ����� ��������� � ����� �������
\xspaceaddexceptions{<< >> ,, ``}

%% ���������� ����� ���
\usepackage{datetime}
% ������������� ������ ����
\ddmmyydate

%% ���������� ����� ��� ����������� ��������� ��������
\usepackage{listings}

\renewcommand{\lstlistingname}{�������}
% ������ ����� ����� ������ ����� ���������
\makeatletter
  \AtBeginDocument{\renewcommand \thelstlisting{\thechapter.\@arabic\c@lstlisting\autodot}}
\makeatother
% ����������� ����� ��-���������


\DeclareNewTOC[
  type = codebox,
  types = codeboxes,
  float,
  floatpos = h,
  floattype = 4,
  nonfloat,
  name = �������,
  listname = {������ ���������}
]{lor}
\setuptoc{lor}{chapteratlist}
\makeatletter
  \AtBeginDocument{\renewcommand \thecodebox{\thechapter.\@arabic\c@codebox}}
\makeatother

%% ���������� ����� ��� �������������� ����������
\usepackage[noend]{algorithmic}
\algsetup{
  linenosize = \tiny,
  linenodelimiter = \ 
}
% ������������ �������� �����
\def\algorithmicrequire{\textbf{����:}}
\def\algorithmicensure{\textbf{�����:}}
\def\algorithmicif{\textbf{����}}
\def\algorithmicthen{\textbf{��}}
\def\algorithmicelse{\textbf{�����}}
\def\algorithmicelsif{\textbf{����� ����}}
\def\algorithmicfor{\textbf{���}}
\def\algorithmicforall{\textbf{��� ����}}
\def\algorithmicdo{}
\def\algorithmicwhile{\textbf{����}}
\def\algorithmicrepeat{\textbf{���������}}
\def\algorithmicuntil{\textbf{����}}
\def\algorithmicloop{\textbf{����}}
\def\algorithmiccomment#1{\quad/\!/ \textit{#1}}

%% ����������� ����� ��� ����������� �����������
\usepackage{hyperref}
\urlstyle{sf}

% ������ ���������� ������� �������� �����
\makeatletter
\g@addto@macro{\Url@acthash}{\Url@Edit\Url@String{//}{/\kern-0.2em/}}
\makeatother

% ������� �������������� ���������� �����
\usepackage{mathrsfs}

% ���������� �����
\usepackage{yfonts}

% ���������
\usepackage{pb-diagram}

% ��� ������������� ������������
\usepackage{verbatim}

%% ������ ������� �������� (�������, ��������, ������, ���. �������) � ���������
\usepackage{totcount, etoolbox}
\regtotcounter{page}
\regtotcounter{figure}
\regtotcounter{table}
\regtotcounter{codebox}
\regtotcounter{enumiv}

\newcounter{totfigures}
\newcounter{tottables}
\newcounter{totlistings}

\providecommand\totfig{} 
\providecommand\tottab{}
\providecommand\totlst{}

\makeatletter
\AtEndDocument{%
  \addtocounter{totfigures}{\value{figure}}%
  \addtocounter{tottables}{\value{table}}%
  \addtocounter{totlistings}{\value{codebox}}%
  \immediate\write\@mainaux{%
    \string\gdef\string\totfig{\number\value{totfigures}}%
    \string\gdef\string\tottab{\number\value{tottables}}%
    \string\gdef\string\totlst{\number\value{totlistings}}%
  }%
}
\makeatother

\pretocmd{\chapter}{\addtocounter{totfigures}{\value{figure}}\setcounter{figure}{0}}{}{}
\pretocmd{\chapter}{\addtocounter{tottables}{\value{table}}\setcounter{table}{0}}{}{}
\pretocmd{\chapter}{\addtocounter{totlistings}{\value{codebox}}\setcounter{codebox}{0}}{}{} 

% ����� ��� �������� ���������� ����������
\usepackage{makeidx}

% ����� ��� ���������
\usepackage{epigraph}

% ����� ��� ������ �� ��������� ��������
\usepackage{lastpage}

% ������� � ������ ��� ������
\usepackage{textcomp}

% Subsection ��� ���������
\makeatletter
\newcommand*\addsubsec{\secdef\@addsubsec\@saddsubsec}
\newcommand*{\@addsubsec}{}
\def\@addsubsec[#1]#2{\subsection*{#2}\addcontentsline{toc}{subsection}{#1}
  \if@twoside\ifx\@mkboth\markboth\markright{#1}\fi\fi
}
\newcommand*{\@saddsubsec}[1]{\subsection*{#1}\@mkboth{}{}}
\makeatother

% Subsubsection ��� ���������
\makeatletter
\newcommand*\addsubsubsec{\secdef\@addsubsubsec\@saddsubsubsec}
\newcommand*{\@addsubsubsec}{}
\def\@addsubsubsec[#1]#2{\subsubsection*{#2}\addcontentsline{toc}{subsubsection}{#1}
  \if@twoside\ifx\@mkboth\markboth\markright{#1}\fi\fi
}
\newcommand*{\@saddsubsubsec}[1]{\subsubsection*{#1}\@mkboth{}{}}
\makeatother

%%%%%%%%%%%%%%%%%%%%%%%%%%%%%%%%%%%%%%%%%%%%%%%%%%%%%%%%%%%%%%%%%%%%%
%% ������ ���������� �������� � �������������� ��������, �����
%% ���� �������� ����������� �� ����� ������ 
%%
%% ��� ������� �� ���� ������ 'russmath', �����: leontiev@aport.ru
%%
\newif\ifBrkFlag
\def\allowmathbreak{\global\BrkFlagtrue\global\relpenalty=10000\global\binoppenalty=10000}
\def\cancelmathbreak{\global\BrkFlagfalse\global\relpenalty=500\global\binoppenalty=700}
\allowmathbreak

\makeatletter
\def\m@th{\mathsurround=0pt}
\def\@thick{\hbox{\m@th$\mskip\thickmuskip$}}
\def\@med{\hbox{\m@th$\mskip\medmuskip$}}

\def\brel#1{\ifBrkFlag\discretionary{\@thick\hbox{\m@th$#1$}}{}{}\fi #1}
\def\bbin#1{\ifBrkFlag\discretionary{\@med\hbox{\m@th$#1$}}{}{}\fi #1}
\makeatother
%%%%%%%%%%%%%%%%%%%%%%%%%%%%%%%%%%%%%%%%%%%%%%%%%%%%%%%%%%%%%%%%%%%%%
%% МАКРОСЫ
%%%%%%%%%%%%%%%%%%%%%%%%%%%%%%%%%%%%%%%%%%%%%%%%%%%%%%%%%%%%%%%%%%%%%
%%
%%

%% Текстовые сокращения
% Exapmli gratia (eg.) = Например (напр.)
\newcommand{\eg}{например,\xspace}
\newcommand{\Eg}{Например,\xspace}
% Et cetera (etc.) = И так далее (и т.д.)
\renewcommand{\etc}{и~т.\,д.\xspace}
% Id est (i.e.) = То есть (т.е.)
\newcommand{\ie}{т.\,е.\xspace}
\newcommand{\Ie}{Т.\,е.\xspace}
% Et alii (et al.) = И другие (и др.)
\newcommand{\etal}{и~др.\xspace}
%% (стр. 99)
\newcommand{\pagerefrus}[1]{(стр.\,\pageref{#1})}

%% Макросы для заглушек
\newcommand{\fixme}{FIXME}
\newcommand{\unref}{[???]}

\newcommand{\kips}{$\text{кк}/\text{с}$\xspace}

%% Макросы
%% Для выделения текста
\newcommand{\strong}[1]{\textbf{#1}}
% Для вставки иностранных слов и словосочетаний
\newcommand{\foreigne}[1]{\mbox{англ.}~\emph{\foreignlanguage{english}{#1}}} % английский
\newcommand{\foreigng}[1]{\mbox{греч.}~\foreignlanguage{greek}{#1}} % греческий
\newcommand{\foreignl}[1]{\mbox{лат.}~\emph{\foreignlanguage{latin}{#1}}} % латинский
\newcommand{\foreignfrak}[1]{\mbox{нем.}~\textfrak{#1}} % немецкий фрактурный
%% Для выделения ключевых слов (терминов)
\newcommand{\keyword}[2][russian]{\textsf{\foreignlanguage{#1}{#2}}}
\newcommand{\keyworde}[2]{\textsf{#1} (\foreigne{#2})}
\newcommand{\keywordg}[2]{\textsf{#1} (\foreigng{#2})}
\newcommand{\keywordl}[2]{\textsf{#1} (\foreignl{#2})}
\newcommand{\keywordfrak}[2]{\textsf{#1} (\foreignfrak{#2})}

%% Для вставки сокращений
\newcommand{\gap}{\ensuremath{\langle\ldots\rangle}\xspace}
%% Для вставки ссылки на рисунки и таблицы
\newcommand{\seefig}[1]{см.~рис.~\ref{#1}}
\newcommand{\fig}[1]{рис.~\ref{#1}}
\newcommand{\seetab}[1]{см.~табл.~\ref{#1}}
\newcommand{\tab}[1]{табл.~\ref{#1}}

%% Окружение для вставки длинных цитат на различных языках
\newenvironment{mquote}[1][russian]{%
  \bigskip%
  \begin{addmargin}{1.1\parindent}%
  \setlength{\parindent}{0pt}%
  \setlength{\parskip}{\smallskipamount}%
  \selectlanguage{#1}%
  \sffamily\small%
}{
  \end{addmargin}%
  \bigskip%
}

%% Более красивый вариант написания названия языка C++
\newcommand{\CPP}{C\kern-.05em\raise.23ex\hbox{+\kern-.05em+}\xspace}
\newcommand{\CCPP}{C/C\kern-.05em\raise.23ex\hbox{+\kern-.05em+}\xspace}

\newcommand{\midtilde}{\raise-0.5ex\hbox{\textasciitilde}}

\renewcommand{\land}{\bbin{\mathbin{\&}}}
\renewcommand{\lor}{\bbin{\mathbin{\vee}}}
%% Импликация
\newcommand{\impl}{\bbin{\mathbin{\Rightarrow}}}
%% Эквивалентость
\newcommand{\equi}{\bbin{\mathbin{\Leftrightarrow}}}
%% Обратная импликация (converse implication)
\newcommand{\coimpl}{\bbin{\mathbin{\Leftarrow}}}

\newcommand{\bin}{\brel{\in}}
\newcommand{\nin}{\brel{\notin}}
%% Значок пустого множество (empty set)
\newcommand{\es}{\varnothing}
%% Равенство 
\newcommand{\eq}{\brel{=}}
%% Неравенство 
\renewcommand{\neq}{\brel{\ne}}
%% Больше или равно (greater or equal)
\renewcommand{\ge}{\brel{>}}
\renewcommand{\geq}{\brel{\geqslant}}
%% Меньше или равно (less or equal)
\renewcommand{\le}{\brel{<}}
\renewcommand{\leq}{\brel{\leqslant}}
%% Строгое подмножество
\newcommand{\subs}{\brel{\subset}}
%% Подмножество
\newcommand{\subseq}{\brel{\subseteq}}
%% Разность (without)
\newcommand{\wo}{\bbin{\setminus}}
%% Симметрическая разность
\newcommand{\swo}{\bbin{\mathbin{\div}}}
%% Декартово произведение
\newcommand{\dec}{\bbin{\times}}
%% Запретитель разрыва страницы между двоеточием и списком
\makeatletter
\newcommand{\listnopagebreak}{\par\nobreak\@afterheading}
\makeatother
%% До и после нашей эры
\newcommand{\bc}{до\,н.\,э.}
\newcommand{\ad}{н.\,э.}

%% ����� natbib ������������ ��� ��������������� ������� \bibsection,
%% ����� ��������� ������ ����������, �������� ������� multibib, 
%% ���������� �� ����� ������ (\chapter), � ����� �������� (\section)
%% ����� ���� ����� ��������� ���������� ������ cite (sort&compress)
\usepackage[numbers, square, comma, sort&compress]{natbib}
% ������ ������ ��������� ������ ���������� � [1] �� 1.
\makeatletter\renewcommand\@biblabel[1]{#1.}\makeatother
% ������ ����� �� ������
\renewcommand\bibsection{\addsec*{\refname}}

%% ���������� ����� multibib, ����� ������� ������ ���������� �� �����
\usepackage{multibib}
% ���������� �������� � ���� ��� �������� ������ ����������
\newcites{book}{����������}
\newcites{article}{������}
\newcites{diss}{����������� � ������������}
\newcites{online}{������-���������}
\newcites{patent}{�������}

%% ������������� ����� ������ ���������� ���� 7.0.5-2008
\providecommand*{\BibEmph}[1]{\emph{#1}}


% Названия
\newcommand{\worktitle}{Теплопроводность в несжимаемой жидкости}
\newcommand{\worksubject}{Курсовая работа по курсу <<Качественные методы гидродинамики>>}

% Настраиваем параметры документа
\hypersetup{
  unicode,
  colorlinks,
  linkcolor = blue,
  citecolor = red,
  bookmarksopen = true,
  bookmarksnumbered = true,
  bookmarksopenlevel = 1,
  pdftitle = {\worktitle},
  pdfauthor = {Павел Крюков <kryukov@frtk.ru>},
  pdfsubject = {\worksubject},
  pdfkeywords = {гидродинамика, жидкость, несжимаемая, теплопроводность},
  pdfcreator = {LaTeX 2e with TeXnicCenter 2.0},
  pdflang = {ru-RU},
  pdfpagelayout = {SinglePage}
}

\begin{document}

%% Титульный лист
%%%%%%%%%%%%%%%%%%%%%%%%%%%%%%%%%%%%%%%%%%%%%%%%%%%%%%%%%%%%%%%%%%%%%
%% Титульный лист
%%%%%%%%%%%%%%%%%%%%%%%%%%%%%%%%%%%%%%%%%%%%%%%%%%%%%%%%%%%%%%%%%%%%%

%% Изменяем слегка нижнее поле страницы
\newgeometry{left=3cm, right=2cm, top=2cm, bottom=1cm} 
\onehalfspacing

\begin{titlepage}

\begin{center}
  Министерство образования и~науки Российской Федерации\\
  Московский физико"=технический институт (государственный университет)\\
  Факультет радиотехники и~кибернетики\\
  \textsf{Кафедра теоретической физики}

  \begin{flushright}
    \textsf{УДК 536.22}
%    \textit{На правах рукописи}
  \end{flushright}

  \vspace{15mm}  
  {\large Крюков Павел Игоревич}

  \bigskip
  {\LARGE\sffamily\bfseries
    \worktitle\\
  }

  \bigskip
  \textsf{\worksubject}

\end{center}

\vspace*{\fill}
\begin{flushright}
  \newlength{\signline}
  \newlength{\signskip}
  \newcommand{\signtext}{{\footnotesize\sffamily(подпись и дата)}}
  \setlength{\signline}{50mm}
  \setlength{\signskip}{\signline-\widthof{\signtext}}
  \newcommand{\signplace}{\bigskip\rule{\signline}{.6pt}\\[-2mm]\signtext\hspace*{.5\signskip}}

  \bigskip\bigskipНаучный руководитель:      
  \par\smallskip
  д.\,ф.-м.\,н.\\
  Крайнов Владимир Павлович
  \par\signplace
    
  \bigskip\bigskipИсполнитель: 
  \par\smallskip
  студент 916 гр.\\
  Крюков Павел Игоревич
  \par\signplace
  
\end{flushright}

\vspace*{\fill}
\begin{center}
  Москва\\
  2014
\end{center}

\end{titlepage}

%% Восстанавливаем исходную геометрию страницы
\restoregeometry


%% Тело
\begingroup
\let\cleardoublepage\clearpage
\tableofcontents
\endgroup

\addsec{� MIPT\LaTeX}

MIPT\LaTeX~"--- ����� ������������, ������, �������� �~��������, ������� ���~��������� ����� ��~������� ����� ����������~������� �~������ �����.
�~����������� ���������� �~��������� ������ ��������� ������� ��������� �~�������� ������� ����������������� ���������� ����������� ������-������������ ������������:
\begin{itemize}
    \item�.\,�����
    \item�.\,������
    \item�.\,�������
    \item�.\,������
\end{itemize}

C~2015 ���� MIPT\LaTeX~"--- ��������� ����������� �����������, ���������������� ��~�������� MIT.
�������� git-����������� �������� ���~�������� �~�������� ��������� ��~������ \url{https://github.com/pavelkryukov/miptlatex}.

\pagebreak

\addsec{������������� ��������� ��� �������� \LaTeX}

\begin{enumerate}[label = \arabic*.]
    \item��������� �.\,�. \emph{\href{http://www.ihed.ras.ru/subsecond2007/papers/lv3ed.pdf}{����� � ������ � ������� \LaTeX}}~"--- 3-� ���. �.: �����, 2003.~"---448~c.
		\item�������� �.\,�. \emph{\href{http://www.ccas.ru/voron/download/voron05latex.pdf}{\LaTeXe � ��������}}~"--- 2005.
\end{enumerate}

\addsec{��������� � ��������� MiK\TeX}

MIPT\LaTeX{} ������������ ���~������ �~������������� MiKTeX ��� �� Windows.
��� ����� ��������� �~������������ �����: \url{http://miktex.org/downloads}.
����� ��������� ����� ��������� ��������� \emph{MikTeX Options}~(���.~\ref{fig:miktex}), �����~���������� ��������� �������~(1) �~��������� ��������� �������~(2).
�������� ����������� ������� ��������� ������������� ���~������ ������� �����������.

\begin{figure}[h]
    \centering
    \includegraphics[scale = 1.0]{pictures/miktex.png}
    \caption{��������� ��������� \emph{MikTeX Options}}
    \label{fig:miktex}
\end{figure}

\pagebreak

\addsec{��������� � ��������� \TeX{}nicCenter}

\TeX{}nicCenter~"--- ������������� ���� ���~������ ��~Windows ������� �������� \TeX-������, ������� ����� ������������� ���~�~MiK\TeX{}, ���~�~�~�������� �������������, .
��������� ������ ��������� �������� ���~�������� ��~����������� �����: \url{http://www.texniccenter.org/download/}.

����� ��������� ��������� �� ������� ���������� � ���������� ���������� �� ������� �����, ���������� ���������� ��������� �������������.
��� ����� ����� ��������� ������� ��������� �������� (\emph{Build/Define output profiles\ldots} ��� \emph{Alt+F7})~(���.~\ref{fig:profiles}), ������ ������ \emph{Import\dots} (1), ������� ���� ������������ \emph{<workspace>/files/jobfile.tco} �~�������� ���� �~������ (2) ����������~\emph{tools} ��~����������.

\begin{figure}[h]
    \centering
    \includegraphics[scale = 0.7]{pictures/profiles.png}
    \caption{��������� ������� ��������� �������� TeXnicCenter}
    \label{fig:profiles}
\end{figure}

\addsec{������������� ����������� ������}

��������� � �������� �����������, ��~������ ��������� ����������� ��������� �~�������������� ���������������� ������ MIPT\LaTeX.
��������� \LaTeX{} ��������� ��������� ������������� ���� ���~����������, ��~������ ����������� ��������� �~�������������� ������� ����� git ���������� �������� �����; \eg \emph{<workspace>/works/masterthesis/} ���~\emph{<workspace>/articles/ieee/}.
�������� ����������� ������ ���~������������ ���������� ����������� git, ���~���~������������ ��������� ���������� ������������� �������� �~������� �������� ������ �~�������� ���������� MIPT\LaTeX.

���������� � ���������������������� �������� ����� �~������� ����� ���������� ��������� ��~��������� ���������� ���������������� ������, ���������� ���~�����������.
����� �������� ������������ ������, ���������� ����������������� ���������� �~������� BiBTeX, ������������� ��������� �~�~������ ����� (\eg \emph{<workspace>/global.bib}).
�������� ����, ������ �~���������� ����������� ���������������� ������ MIPT\LaTeX, ����� ���� �������� ��~����������� ��������� ����������� MIPT\LaTeX.

\addsec{������� �����}

�������� ��������, ��� �������� ������� � ����������� ������ ������ ����������� ������ ����������� � �������� �������� ������ �������������� �����.
��� ����� � ����������� ��������� 4 �������, ���������� � �������������� MIPT\LaTeX:
\begin{enumerate}
    \item{About~"--- �������� MIPT\LaTeX, ������� �� ������ �������.}
    \item{Kraynov~"--- ������ �������� �� ������������� ������ � ������� ����������� ������.}
    \item{VoIP~"--- ������ �������� � �������������, �������������� � ���������� ������� ����������.}
\end{enumerate}

����� ���������� ��������� ������������ ������������� �������� ���������� ���� ��� ��������, ��������� �~������������ ������ �~������� ������ ����������.

\addsec{���������������� �����}

\addsubsec{config.tex}

���� config.tex �������� �������� �������� ��������� ��������� ��� ������� �������� � ������������ ����������� ����������� �������.

\addsubsec{macros.tex}

���� macros.tex �������� ����������� �������� ��������������� ��������: ����������, ������ ��������, ������������ ��������������, � ����� ��������������� �������������� ��������, ������������ ������� ��������.

\addsubsec{paragraph.tex}

����������� ����� paragraph.tex ��������� �������� ���������� ����� ����������~"--- � �����-�� ������� ��� ����������� ����������.

\addsec{������}

\addsubsec{mips.sty}

����� mips.sty ��������� ��� ��������� ��������� �� ����� ���������� ����������� ������� ������ MIPS.

\addsec{�������}

\addsubsec{trudy.bat}

������ <<trudy.bat>> ������������� ����������� ������ ��� ������� <<����� ����>> � ������, ��������� ���������.
\emph{������ � ����������}

\addsubsec{endline}

������� ����� ��������� ���������� � �������� \TeX ��������� ���������� ������� ����������� �� ��������� ������.
������ <<endline.pl>> ��������� ������������ ���������� ������ ���������� ����� ���� ������������.
� ���������� ������ �� ����� ��������� ������ ����� �����, ��� ��� ������������ ���� ��������.

������ <<endline.sh>> ��������� ������ <<endline.pl>> ��� ���� ������ *.tex, ����������� � ��������� ������ ���������� ����������.

\addsubsec{nobr}

��� �������, ��� ������ ������� �������� ��������� ��� ���������� <<������� ���������>>, \ie ���������, ����������� �� ������ ������� � ����������� ������.
����������� ������������ ��������� ��� �������� �����������, ����� ��� <<�>>, <<��>>, <<��>>; � ����� ������ ����� �������� ���� (<<�����~��>>, <<�����~��>>).
������ <<nobr.pl>> �������� <<�������>> ������� �� <<�����������>> � ����������� ��������� �������.
���������� ����� �������� ��� ��������� �����, ������������ ���� ���������� �� �����.

������ <<nobr.sh>> �������� �������������� ������ �������������� ������ � ����������� <<*.tex>> �� ������������ � ��������� ����������.
��������� ����� ������ ����������� � ���������� � ����������� *.backup.

\emph{������ � ����������}

\addsec{�������� ������ ����������}

\addsec{�������� ����������� ���������}

\addsec{�������� ����������}



%% Приложения
\appendix
% Делаем нумерацию приложений русскими буквами
\renewcommand{\thechapter}{\Asbuk{chapter}}

\end{document}
