\addchap{Введение}

\keyword{IP-телефония}~"--- телефонная связь по~протоколу IP.
Эта технология является приложением более общей технологии \keyworde{VoIP}{voice over IP}, подразумевающей и~другие применения передачи голоса, такие как~телетрансляции, системы оповещения \etc
Потребность в~создании технологий IP-телефонии возникла по~нескольким факторам:\listnopagebreak
\begin{itemize}
    \itemраспространённость протокола IP для~передачи данных и~стремлению к~конвергенции технологий;
		\itemнеобходимость в~преодолении ограничений, накладываемых технологией телефонной сети общего пользования.
\end{itemize}

IP-телефония на~текущий момент реализуется многими протоколами, наиболее распространённые из~них:\listnopagebreak
\begin{itemize}
    \item\ H.232;
		\item\ SIP;
		\item\ MGCP.
\end{itemize}

Цель этого реферата~"--- рассмотреть самые распространённые решения в~IP-телефонии на~базе перечисленных выше протоколов, провести сравнительный анализ, включая анализ рынков.