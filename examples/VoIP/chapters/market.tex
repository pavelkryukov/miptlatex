\chapter{Анализ рынка}

Рынок IP-телефонии~"--- развивающийся рынок IT-услуг.
В~2012 году объём всего рынка в~России составил 4,59 млрд рублей, в~2013 ожидался рост до~5,35 млрд рублей.
17\% населения России (27 млн человек) постоянно используют технологии IP-телефонии, и~с течением времени эта цифра постепенно увеличивается.
Скорость увеличения, однако, ограничена отсутствием необходимости в~сервисах IP-телефонии у~многих абонентов.
Стоит отметить, что~многими экспертами рынок IP-телефонии в~России оценивается как~низкий \citeonline{discovery}.

\section{Частный сегмент}

Услуги IP-телефонии частного сегмента можно разделить на~несколько типов:
\begin{itemize}
    \itemуслуги операторского класса, поставляемые в~пакете с~услугами широкополосного доступа, телевидения \etc;
		\itemуслуги Over-the-Top, поставляемые сторонними компаниями.
\end{itemize}

Как~правило, услуги первого типа предоставляются <<классическими операторами>> связи, такими как~Билайн, Стрим, Акадо и~другие.
В~последнее время, однако, многие пользователи Интернета выбирают IP-телефонию второго типа, а~именно технологии Skype (Microsoft), FaceTime (Apple), GoogleTalk (Google) \etc.
Преимуществом этих технологий для~пользователей является повышенная мобильность, распространённость технологий по~миру~"--- можно звонить в~любую точку мира по~одинаковой цене.
Как~правило, эти технологии несовместимы между~собой, и~поэтому пользователю приходится в~той или~иной мере пользоваться ими всеми.

Технологии сторонних компаний завоёвывают всё большую популярность относительно~услуг ТфОП: так~из-за~этого фактора, например, по~итогам 2012 года ожидаемое снижение доходов европейских сотовых операторов составило от~1,5\% до~4\%.
Наиболее заметное снижение выручки заметно в~секторе дальней связи, где в~России операторы фиксированной связи за~2012 год потеряли около~25\% выручки.
В~связи с~этим, операторы всё больше и~больше уделяют внимание собственным услугам IP-телефонии, а~также снижению цен на~услуги <<классической>> телефонии.

\section{Корпоративный сегмент}

Основная доля российского рынка (73\%) IP-телефонии приходится на~корпоративный сектор, из~которой 45\% составляют услуги виртуальных АТС.
В~этом сегменте, согласно прогнозам J'son \& Partners Consulting, объём рынка к~2016 году увеличится до~3,8 млрд руб, а~среднегодовой темп роста рынка за~период с~2010 по~2016 год составит 30\%.
Соответственно, ожидается снижение доходов операторов фиксированной телефонии как~устаревающей технологии; средняя выручка на~одного пользователя с~использованием новой технологии сравняется с~этим показателем для~фиксированной телефонии в~2015 году \citeonline{jsonmarkets}.

Согласно результатам проведённого исследования \citeonline{jsonmarkets} наибольший спрос на~виртуальные АТС зафиксирован у~компаний малого и~среднего бизнеса, занимающихся ритейлом/продажами, телекоммуникациями, IT и~промышленных компаний; эти компании составляют 45\% рынка услуг виртуальных АТС.
22,5\% рынка составляют следующие сектора крупного бизнеса: телекоммуникации, IT и~промышленные компании, а~так~же государственные компании и~учреждения.